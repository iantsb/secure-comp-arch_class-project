\documentclass[sigconf,authoryear]{acmart}
\settopmatter{printfolios=true}
\usepackage{graphicx}
\usepackage[normalem]{ulem}
\usepackage[T1]{fontenc}       % proper font encoding
\usepackage[utf8]{inputenc}    % UTF-8 input
\usepackage{comment}
\usepackage{enumitem}
\usepackage{todonotes}



\title{Literature Review\protect\\ \large September 2025}
\author{Ian Barnaby}\author{}\author{}

\begin{document}

\maketitle

\section{Introduction}
This project aims to research and design an asymmetric encryption method targeting a processor and AI accelerator pair. In the 
theorized application, the accelerator is taken as a trusted base holding the IP of a neural network. It is therefore desired to
have an encryption scheme between CPU and accelerator to prevent malicious actors from performing snooping attacks on the bus connecting
the accelerator and processor. This project specifically considers accelerators utilizing non-volatile memory to perform compute 
in-memory operations. These types of devices have been demonstrated to allow for information to be encoded in physical device 
properties, without being stored in the current state of the cell. In the proposed encryption mechanism, values encoded 
in the memory cells are used to generate a private/public key pair, used to encrypt processor/accelerator communication.

\section{Relevant Background}
This literature review focuses on literature regarding the relevant NVM devices/device properties, asymmetric encryption, PUF-based key
generation, fault injection defense


\section{Search Terms}
In searching for papers, databases such as EngineeringVillage and IEEEXplore were used. Search terms included terms such as, 
but not limited to:
\begin{itemize}
    \item 
\end{itemize}


\section{Paper Reviews}
\subsection{Hiding Information for Secure and Covert Data Storage in Commercial ReRAM Chips}
This paper demonstrates a technique for covert data storage in ReRAM devices \cite{ReRAM_hiding}. The researchers state that the set/reset time of
a ReRAM cell changes with each write cycle. This occurs due to oxygen vacancies in the oxide layer used to construct the conductive
filament and limits the total number of write cycles the cell can handle before it no longer functions. The researchers have observed
that, once the set/reset characteristics of a cell have been altered, they can reliable determine cells with more or less "wear". By
setting a point at which set/reset times can be read as binary "1" or "0", cells can intentially be worn down to encode values in the
set/reset time. The researchers demonstrate that fresh cells can be worn down in desired bit patterns, and the data can be be reliably
recovered until a threshold is reached. After a certain number of write cycles, the data encoded in the set/reset time becomes too 
noisy. As such, this paper provides the basis for this work of establishing a method of encoding data within ReRAM cells separate from 
the memory values, that changes based on how many writes have been issued to the device.

\subsection{Overview of NVM Devices}
The above paper has proven the potential for data encoding in ReRAM, but other NVM technologies are worthy of review. As NVM and CIM are 
unsolved fields, commercial applications may move away from ReRAM. There are three other types of NVM technologies deserving particular 
attention: ECM, phase-change, and magnetic memory. While other technologies exist, they are substantially less relevant or developed for 
deep learning applications. 

\subsubsection{ECM} 
ECM (electrochemical metallization) memory operates on a similar principle to ReRAM: 
a low resistance state is created by forming a conductive filament between two electrodes, and a high resistance state is created by 
destruction of the filament. In ReRAM (also called valence change memory), this conductive filament is formed by oxygen vacancies 
in an exotic metal oxide layer. In ECM, this conductive filament is formed by the migration of metal ions from an active electrode
towards an inert electrode through an oxide layer. An existing work demonstrates the set/reset characteristics of ECM in regards to 
endurance \cite{ECM}, supporting that set/reset characteristics change over many write cycles. While further study would be required to 
determine if the same technique can be reliably applied to ECM, the required behavior is present.

\subsubsection{PCM}
PCM (phase change memory) differs in that it does not utilize a conductive filament. In PCM, a layer of thin chalcogenide is deposited 
between two conducting contacts. By applying a moderate current pulse, the chalcogenide layer is heated, forming a conducting 
crystalline layer, creating a low resistance state. By applying a short, high current pulse, the chalcogenide layer is melted, 
rapidly cooling into a non-conducting amorphous layer, creating a high resistance state. An existing work demonstrates, similarly to ECM, 
that the set/reset characteristics change over time, studied in terms of device endurance \cite{PCM}. Again, further study would be 
required to determine if the same technique can be applied, but the required behavior is present.

\subsubsection{STT-MRAM}
STT-MRAM (spin transfer torque magnetic RAM) operates on a substantially different principle than any of the above technologies. 
An STT-MRAM cell consists of a reference magnetic layer (magnetization direction is fixed), free magnetic layer (magnetization direction 
is variable) and a tunneling barrier between the two. The state encoding is determined by the direction of an applied spin-polarized 
current. Applied in one direction, spin-polarized electrons align the free layer to the reference layer, creating a low resistance state. 
Applied in the opposite direction, the free layer is forced to align opposite to the reference layer, creating a high resistance state. 
Existing works studying the endurance of STT-MRAM in deep learning applications demonstrate devices capable of >$10^{12}$ write cycles 
with little to no timing drift \cite{MRAM}. As MRAM does not rely on a filament creation or phase change, only magnetization direction, 
this makes intuitive sense. Unlike the previous technologies, STT-MRAM does not display the required behavior, and the encoding technique 
is unlikely work.


\subsection{Paper 3 - Ryn}


\subsection{Paper 4 - Ryn } 

\subsection{Bus Encryption for Low-Power Systems}
This paper, \textit{Sealer: In-SRAM AES for High-Performance and Low-Overhead Memory Encryption} \cite{Sealer}, presents a lightweight
approach to securing memory and bus transactions in low-power systems. The authors note that conventional bus and memory encryption
schemes often rely on AES engines placed in the memory controller, which introduce significant latency and energy overhead since
encryption/decryption lies on the critical path of every memory access. To address this, the Sealer architecture repurposes SRAM
subarrays to perform AES operations directly within the memory array, exploiting intrinsic bitline-level parallelism. This allows
data to be encrypted before leaving the chip and decrypted upon return, thereby protecting against bus snooping and cold-boot attacks
without the heavy cost of dedicated crypto hardware. The results demonstrate up to two orders of magnitude improvement in throughput-per-area
and a 3$\times$ reduction in energy compared to prior solutions. For this project, the relevance lies in the demonstration that
bus encryption can be achieved with minimal hardware overhead, suggesting that similar lightweight encryption approaches
could be adapted for CPU-to-accelerator communication.

\subsection{PUF-Based Encryption Using ReRAM Devices}
The paper \textit{An Error Correction Approach to Memristors PUF-based Key Encapsulation} \cite{ReRAM_PUF} explores the use of
ReRAM-based Physically Unclonable Functions (PUFs) for secure key generation and encryption. The authors propose a keyless
encapsulation protocol that leverages the inherent resistance variability of ReRAM cells to generate unique device responses,
which are then used directly for message encryption. Unlike traditional schemes, this avoids the need for stored cryptographic
keys, reducing vulnerability to key extraction attacks. However, because ReRAM PUF responses are sensitive to environmental
conditions and device noise, the paper introduces error correction coding (ECC) mechanisms (Reed-Solomon and BCH codes) to
stabilize the responses and ensure reliable decryption. Experimental results confirm that the scheme can produce noise-free
messages under typical operating variations. For this project, the significance is clear: ReRAM-based PUFs provide a natural
hardware root of trust that can generate evolving keys tied to device physics, aligning directly with the proposed idea of
leveraging NVM wear characteristics for asymmetric encryption in IMC accelerators.

\section{Summary of Field}
\begin{comment}
    - field summary
    - paragraph
\end{comment}


\bibliographystyle{ACM-Reference-Format}
\bibliography{main}
\end{document}

